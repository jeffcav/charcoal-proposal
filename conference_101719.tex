\documentclass[conference]{IEEEtran}
\IEEEoverridecommandlockouts
% The preceding line is only needed to identify funding in the first footnote. If that is unneeded, please comment it out.
\usepackage{cite}
\usepackage{amsmath,amssymb,amsfonts}
\usepackage{algorithmic}
\usepackage{graphicx}
\usepackage{textcomp}
\usepackage{xcolor}
\def\BibTeX{{\rm B\kern-.05em{\sc i\kern-.025em b}\kern-.08em
    T\kern-.1667em\lower.7ex\hbox{E}\kern-.125emX}}
\begin{document}

\title{A Scallable Microarchitecture for \\ eBPF-based Packet Filter Offload\\
{\large Project Proposal}
}

\author{\IEEEauthorblockN{Jefferson Cavalcante}
\IEEEauthorblockA{\textit{MDCC} \\
\textit{Universidade Federal do Ceará}\\
Fortaleza, Brazil \\
jeff.cav@alu.ufc.br}
}

\maketitle

\begin{abstract}
TODO
\end{abstract}

\begin{IEEEkeywords}
component, formatting, style, styling, insert
\end{IEEEkeywords}

\section{Introduction}
Nowadays' Data Centers servers are able to leverage thousands of client applications with the use of containers, a lightweight Operating-System-level virtualization technique [citar From laptop to Lambda]. In such a scenario, it is ideal that the expensive CPU cores on Data Centers servers spend most of their time running client applications to maximize revenue, also avoid having idle cores that lead to expensive resource underutilization [citar dCCPI-Predictor]. 

However, having non-underutilized cores doesn't necessarily mean higher revenue per core. For example, in [cite Throughput and Latency of Virtual Switching] authors show how virtual network switching can increase CPU load, also reduce throughput and introduce latency and jitter to virtual networks, impacting applications performance and reducing revenue per core of the data center, although such expensive cores are busy but not processing paying-clients' applications.

Developers from Suse presented in 2020, at the FOSDEM conference, how one of the most used tools for containers orchestration, Kubernetes, make heavy use of Linux netfilter engine for load balancing and firewalling, which in turn leads to prohibitive numbers in terms of rules load and processing time required for packet filtering as the number of applications increases to dozens of thousands [cite Usenix Cilium Fosdem].

For scenarios in which there's the need for high number of packet filtering rules, extended Berckeley Packet Filter (eBPF) programs are a viable solution. These programs can be responsible for packet processing and filtering in a very efficient way when compared to how the Linux kernel does the same with netfilter [cite Fast Packet Processing with eBPF and XDP: Concepts, Code,
Challenges, and Applications; cite Creating Complex Network Services with eBPF:
Experience and Lessons Learned].

Although more efficient, at higher data rates this solution still overloads expensive server cores with packet filtering programs for load balancing, firewalling and intrusion detection for example, which are non-paying-clients applications and reduces data center revenue per core. To address this problem, Netronome built support for offloading eBPF programs to cheaper cores in their smartNICs, and presented a tutorial on the prestigious SIGCOMM conference in 2018 on how to leverage their solution to offload a load balancer written in eBPF [cite netronome smartnic folder].At the time their solution was still limited and not scallable for multiple eBPF programs and did not integrate with their Single-Root I/O Virtualization (SR-IOV) solution, a technology that allows hardware virtualization of PCIe devices and is widely used by network cards to virtualize network interfaces for exclusive use by virtual machines and containers.

\section{Problem Statement}
To support the needs of modern Data Centers workloads and maximize their revenue per CPU core, one of the problems to be addressed is the minimization of packet filter processing on these cores by offloading this task to purpose-specific hardware in a scallable and efficient way.

\section{Related Work}
\label{sec:rel-works}

In recent years, a series of works on purpose-specific hardware to increase energy or time efficiency of specific tasks has emerged, achieving exciting results when compared to when such tasks run on general purpose CPUs. For example, in 2016 a computer architecture which organizes processing units into tiles with near memory for faster neural network processing, the ISAAC \cite{shafiee2016isaac} architecture was proposed to bring together near memory processing and a memristor-based dot product engine, which allows high energy efficiency in matrix-vector multiplications by performing then in the analog domain. More recently, in 2019, ISAAC was evolved into PUMA, with more complex purpose-specific cores (processing units) and tiles, an Instruction Set Architecture (ISA) and a compiler to convert high-level Deep Neural Networks implementation into the PUMA ISA, bringing programmability and generality for execution of a wide range of neural network architectures \cite{ankit2019puma}.

Likewise, recent advances in the development of smartNICs brought the possibility of offloading part of eBPF processing required by modern data center workloads, although current solutions are still very limited and prevent Data Centers from benefiting from their packet filter offloading capabilities [cite Introducing smartNICs in Server-Based Data Plane Processing].

Very recently, however, researchers addressing the problem of reducing jitter of very small applications running in data centers, called lambda-applications, have proposed an approach we can be inspired by. The authors of [cite 2020 smartnic p2p to gpu] proposed that smartNICs, instead of running lambda applications in their cores as proposed before in [cite 2 previous works on offloadind applications to the smartnics], forward data received from the network to be processed by a GPU via Peer-to-Peer (P2P) PCIe communication, without the need for operating system manipulation. The results are then sent back to the smartNIC and forwarded to the network.

\section{Proposal}
Inspired by the works in Section \ref{sec:rel-works}, we propose a research for the development of a high level scallable micro architecture, as \cite{ankit2019puma}, able to run multiple eBPF programs in parallel and to communicate with a network card to offload the packet filer processing from the CPUs. The network card should be able to send ingress packets to eBPF offload cards via P2P before they reach the Operating System, integrating to its SR-IOV implementation for enhanced support for virtualization.

\section*{References}

\begin{thebibliography}{00}
\bibitem{b1} G. Eason, B. Noble, and I. N. Sneddon, ``On certain integrals of Lipschitz-Hankel type involving products of Bessel functions,'' Phil. Trans. Roy. Soc. London, vol. A247, pp. 529--551, April 1955.
\bibitem{b2} J. Clerk Maxwell, A Treatise on Electricity and Magnetism, 3rd ed., vol. 2. Oxford: Clarendon, 1892, pp.68--73.
\bibitem{b3} I. S. Jacobs and C. P. Bean, ``Fine particles, thin films and exchange anisotropy,'' in Magnetism, vol. III, G. T. Rado and H. Suhl, Eds. New York: Academic, 1963, pp. 271--350.
\bibitem{b4} K. Elissa, ``Title of paper if known,'' unpublished.
\bibitem{b5} R. Nicole, ``Title of paper with only first word capitalized,'' J. Name Stand. Abbrev., in press.
\bibitem{b6} Y. Yorozu, M. Hirano, K. Oka, and Y. Tagawa, ``Electron spectroscopy studies on magneto-optical media and plastic substrate interface,'' IEEE Transl. J. Magn. Japan, vol. 2, pp. 740--741, August 1987 [Digests 9th Annual Conf. Magnetics Japan, p. 301, 1982].
\bibitem{b7} M. Young, The Technical Writer's Handbook. Mill Valley, CA: University Science, 1989.
\end{thebibliography}
\vspace{12pt}

\end{document}
